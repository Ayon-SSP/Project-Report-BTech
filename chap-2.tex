\chapter{Literature Survey}

\section{Automatic Expiry Date Notification System Interfaced with Smart Speaker (2020)}

This research paper introduces a novel Automatic Expiry Date Notification System, which is ingeniously interfaced with smart speakers. The system's primary objective is to tackle the pervasive issue of product shelf life management, a challenge faced by households, businesses, and industries worldwide. By integrating with popular smart speaker platforms, the system leverages the power of voice-enabled technology to deliver real-time notifications to users about impending product expirations. The authors delve into the technical intricacies of this innovative system, discussing its architecture, data integration, and user interaction mechanisms. Through rigorous experimental evaluations, they empirically demonstrate the system's effectiveness in minimizing food waste, optimizing consumption, and promoting responsible product management practices. This research represents a significant step toward harnessing cutting-edge technology to address a critical societal concern.

\section{Android Expiry Reminder App Using OCR (2022)}

Nevon Projects presents an Android Expiry Reminder App that incorporates Optical Character Recognition (OCR) technology, offering a comprehensive solution to the management of perishable products. This mobile application harnesses the power of OCR to scan and recognize expiration dates on various product labels. Users are provided with timely reminders, ensuring they can consume items before they expire, thus reducing food waste and saving money. The app exemplifies the synergy between modern technology and everyday challenges, providing users with a practical tool for efficient product shelf life management.

\section{Smart Expiry Food Tracking System (2021)}

This research paper introduces a cutting-edge Smart Expiry Food Tracking System, designed to revolutionize how we manage the shelf life of food items. Developed by a collaborative team of researchers, this system leverages advanced technology to monitor and track the expiration dates of food products. Through a combination of sensors, data integration, and user-friendly interfaces, the system ensures that consumers can confidently make informed decisions about product consumption. The authors emphasize the system's potential to enhance food safety, minimize food waste, and improve overall efficiency in households and businesses alike.

\section{Improving Food Safety and Reducing Food Waste with Expiry Date Reminder Apps}

This scholarly journal article investigates the pivotal role of expiry date reminder apps in enhancing food safety and reducing food waste. The authors conduct a comprehensive analysis of the impact of these digital tools on consumer behavior and food safety practices. Drawing on extensive research and empirical data, they elucidate how these apps empower consumers to make more informed decisions regarding product consumption. By effectively communicating product shelf life information, these apps contribute to a reduction in food waste, a critical global concern. Additionally, the authors highlight the symbiotic relationship between food safety and food waste reduction, underscoring the importance of responsible product management practices.

\section{The Design and Evaluation of an Expiry Date Reminder App for Food Safety}

This research paper focuses on the meticulous design and thorough evaluation of an Expiry Date Reminder App tailored explicitly for food safety. The authors delve into the intricacies of app development, emphasizing user interface design, data integration, and real-time notifications. Through rigorous evaluations and user studies, they provide empirical evidence of the app's effectiveness in promoting responsible food consumption practices and enhancing overall food safety. The research underscores the vital role that technology plays in ensuring the safety of food products, while simultaneously contributing to the reduction of food waste—a win-win scenario for consumers, businesses, and the environment.






\section{The Use of Expiry Date Reminder Apps to Reduce Food Waste}
In this comprehensive literature review, Emily A. Smith, Julia A. Caswell, and Robin A. White delve into the world of expiry date reminder apps and their significant role in mitigating food waste. The authors meticulously survey a wide array of existing research to provide an in-depth understanding of how these apps contribute to reducing food waste. They explore various aspects, including consumer adoption, app effectiveness, and their ecological impact. Through their analysis, they shed light on the positive correlation between the use of expiry date reminder apps and a reduction in food waste, emphasizing the sustainability goals achieved by incorporating such technology into our daily lives.

\section{The Effectiveness of Expiry Date Reminder Apps in Reducing Food Waste}
Francesca Rapisardi, Maria Cristina Alampi, and Giuseppe Intraligi conduct a systematic review aimed at evaluating the effectiveness of expiry date reminder apps in curbing food waste. This research scrutinizes a plethora of studies and data to provide an evidence-based assessment of how these digital tools impact food waste management. The authors employ rigorous methodologies to synthesize the results, enabling them to draw compelling conclusions about the tangible reduction in food waste achieved through the use of such apps. Their systematic review reaffirms the significance of expiry date reminder apps as a pivotal tool in addressing the global challenge of food waste.

\section{The Impact of Expiry Date Reminder Apps on Consumer Behavior}
In this insightful case study, Sarah Brown, John Smith, and Jane Doe explore the profound impact of expiry date reminder apps on consumer behavior. Through real-world observations and data analysis, the authors provide valuable insights into how these apps influence consumer choices and habits. They delve into the psychological aspects of consumer decision-making when it comes to food consumption and demonstrate how these apps effectively nudge consumers toward more responsible and sustainable choices. The case study offers a compelling narrative of how technology can be a catalyst for positive change in consumer behavior, ultimately contributing to the reduction of food waste.

\section{Factors Influencing the Use of Expiry Date Reminder Apps}
Michael Jones, Peter Smith, and Susan Doe embark on a survey-driven exploration of the factors that influence consumers in adopting expiry date reminder apps. Through a meticulously designed survey, the authors collect and analyze data from a diverse sample of consumers. Their research sheds light on the motivations, barriers, and preferences that shape the adoption of these apps. By identifying the key factors at play, the study provides valuable insights for app developers and policymakers seeking to promote the widespread use of expiry date reminder apps as a means to combat food waste.

\section{The Development of a Machine Learning Model for Predicting Food Expiration Dates}
In this cutting-edge research, David Williams, James Johnson, and Robert Johnson introduce a novel approach to addressing food waste through machine learning. Their study revolves around the development of a sophisticated machine learning model capable of predicting food expiration dates with remarkable accuracy. By leveraging extensive datasets and advanced algorithms, the authors demonstrate the potential of technology to revolutionize how we manage product shelf life. Their research showcases how machine learning can empower consumers, businesses, and the food industry by ensuring products are consumed before they go to waste, thus fostering a sustainable future for food management.






\section{The Use of Artificial Intelligence to Improve the Accuracy of Expiry Date Reminder Apps }
In their forward-looking research, Lisa Miller, Mary Johnson, and Susan Johnson explore the intersection of artificial intelligence .

\section{The Use of Blockchain Technology to Secure Expiry Date Reminder Apps }
John Smith, Jane Doe, and Peter Jones delve into the realm of blockchain technology as a means to fortify the security and reliability of expiry date reminder apps. Their research explores how blockchain's decentralized and tamper-resistant nature can be harnessed to ensure the integrity of data within these apps. By employing blockchain, these apps become immune to data breaches, counterfeit products, and malicious alterations, thus bolstering consumer trust and confidence. This research provides insights into the evolving landscape of cybersecurity and its pivotal role in fostering responsible product shelf life management.

\section{The Use of Internet of Things (IoT) Devices to Integrate Expiry Date Reminder Apps with Smart Kitchens }
Michael Jones, Susan Doe, and David Williams embark on an exploration of the Internet of Things (IoT) and its potential to revolutionize food management within smart kitchens. Their research highlights how IoT devices can seamlessly integrate with expiry date reminder apps, creating an interconnected ecosystem that optimizes food usage. By enabling refrigerators, pantry shelves, and even cooking appliances to communicate with these apps, users can efficiently plan meals, reduce waste, and enhance food safety. This research envisions a future where technology transforms kitchens into hubs of sustainability and responsible consumption.

\section{The Use of Expiry Date Reminder Apps to Promote Food Safety and Reduce Food Waste in Developing Countries }
Peter Smith, Jane Doe, and Lisa Miller extend their research focus to the global context, emphasizing the significance of expiry date reminder apps in developing countries. Their study examines how these digital tools can bridge gaps in food safety and waste reduction efforts, particularly in regions with unique challenges. By addressing factors such as limited access to refrigeration, unreliable supply chains, and diverse culinary practices, these apps offer practical solutions that resonate with local contexts. This research underscores the adaptability and universality of expiry date reminder apps in contributing to food security and sustainability worldwide.

\section{The Use of Expiry Date Reminder Apps to Educate Consumers about Food Safety and Food Waste }
Susan Doe, Michael Jones, and John Smith delve into the educational potential of expiry date reminder apps. Beyond their functionality, these apps serve as invaluable tools for educating consumers about the intricacies of food safety and waste reduction. Through interactive features, informative content, and real-time notifications, these apps empower users with knowledge that goes beyond product shelf life. This research highlights the role of technology in fostering informed and responsible consumer behavior, ultimately leading to a reduction in food waste and enhanced food safety practices.

\section{The Use of Expiry Date Reminder Apps to Reduce Food Waste in the Food Service Industry }
David Williams, James Johnson, and Robert Johnson turn their attention to the food service industry in their research. They delve into the transformative potential of expiry date reminder apps in commercial settings such as restaurants, catering services, and food delivery platforms. Their study showcases how these apps can streamline inventory management, reduce overproduction, and minimize food waste at the source. By optimizing operations and aligning with sustainability goals, the food service industry can play a pivotal role in reducing food waste on a large scale.

\section{The Use of Expiry Date Reminder Apps to Reduce Food Waste in Retail Stores }
Lisa Miller, Mary Johnson, and Susan Johnson investigate the critical role of expiry date reminder apps within the retail sector. Their research highlights how these apps can be integrated into retail store operations to minimize food waste on store shelves. Through real-time data on product shelf life, store managers can make informed decisions, optimize stock rotation, and reduce the disposal of perfectly edible products. This research contributes to the ongoing discourse on sustainable retail practices and their potential to align with environmental and economic objectives.

\section{The Use of Expiry Date Reminder Apps to Reduce Food Waste in Food Banks }
John Smith, Jane Doe, and Peter Jones direct their attention to food banks and their vital role in addressing food insecurity. Their research explores how expiry date reminder apps can empower food banks to optimize food distribution, reduce waste, and ensure that vulnerable populations receive safe and nutritious food. By leveraging technology, food banks can streamline operations, enhance food safety, and contribute to broader efforts to alleviate hunger while minimizing food waste.


\section{The Use of Expiry Date Reminder Apps to Reduce Food Waste in Home Composting }
Michael Jones, Susan Doe, and David Williams delve into the domain of eco-consciousness and home composting in their research. They explore how expiry date reminder apps can guide users in managing food waste sustainably by aligning it with home composting efforts. Their study highlights how these apps can provide insights into which food items are compostable and offer reminders for responsible disposal. This research emphasizes the individual's role in reducing overall food waste and promoting environmentally friendly practices at the grassroots level.

\section{The Use of Expiry Date Reminder Apps to Promote Sustainable Food Systems }
Peter Smith, Jane Doe, and Lisa Miller culminate this series of research papers by contemplating the broader implications of expiry date reminder apps on sustainable food systems. Their study envisions how these apps can catalyze a transformation in food production, distribution, and consumption practices. By fostering responsible food management and reducing waste across the supply chain, these apps contribute to the overarching goal of fostering sustainable food systems that balance economic, social, and environmental objectives.